\documentclass[letterpaper,11pt,twoside,]{pinp}

%% Some pieces required from the pandoc template
\providecommand{\tightlist}{%
  \setlength{\itemsep}{0pt}\setlength{\parskip}{0pt}}

% Use the lineno option to display guide line numbers if required.
% Note that the use of elements such as single-column equations
% may affect the guide line number alignment.

\usepackage[T1]{fontenc}
\usepackage[utf8]{inputenc}
\usepackage{longtable}

% pinp change: the geometry package layout settings need to be set here, not in pinp.cls
\geometry{layoutsize={0.95588\paperwidth,0.98864\paperheight},%
  layouthoffset=0.02206\paperwidth, layoutvoffset=0.00568\paperheight}

\definecolor{pinpblue}{HTML}{185FAF}  % imagecolorpicker on blue for new R logo
\definecolor{pnasbluetext}{RGB}{101,0,0} %



\title{Lab 005 - Inference for means}

\author[a]{EPIB607 - Inferential Statistics}

  \affil[a]{McGill University}

\setcounter{secnumdepth}{5}

% Please give the surname of the lead author for the running footer
\leadauthor{Bhatnagar and Hanley}

% Keywords are not mandatory, but authors are strongly encouraged to provide them. If provided, please include two to five keywords, separated by the pipe symbol, e.g:
 \keywords{  Sampling distribution |  Standard error |  Normal
distribution |  Quantiles |  Percentiles |  Z-scores  }  

\begin{abstract}
In this exercise you will practice calculating confidence intervals
using the t-distribution and the bootstrap.
\end{abstract}

\dates{This version was compiled on \today} 

% initially we use doi so keep for backwards compatibility
\doifooter{\url{https://sahirbhatnagar.com/EPIB607/}}
% new name is doi_footer

\pinpfootercontents{in-class exercise on confidence intervals}

\begin{document}

% Optional adjustment to line up main text (after abstract) of first page with line numbers, when using both lineno and twocolumn options.
% You should only change this length when you've finalised the article contents.
\verticaladjustment{-2pt}

\maketitle
\thispagestyle{firststyle}
\ifthenelse{\boolean{shortarticle}}{\ifthenelse{\boolean{singlecolumn}}{\abscontentformatted}{\abscontent}}{}

% If your first paragraph (i.e. with the \dropcap) contains a list environment (quote, quotation, theorem, definition, enumerate, itemize...), the line after the list may have some extra indentation. If this is the case, add \parshape=0 to the end of the list environment.


\begin{longtable}[]{@{}ll@{}}
\toprule
R Code & Value \\
\midrule
\endhead
\texttt{qnorm(p\ =\ c(0.05,\ 0.95))} & -1.64, 1.64 \\
\texttt{qnorm(p\ =\ c(0.025,\ 0.975))} & -1.96, 1.96 \\
\texttt{qnorm(p\ =\ c(0.005,\ 0.995))} & -2.58, 2.58 \\
\texttt{qt(p\ =\ c(0.025,\ 0.975),\ df\ =\ 400-1)} & -1.97, 1.97 \\
\texttt{qt(p\ =\ c(0.025,\ 0.975),\ df\ =\ 25-1)} & -2.06, 2.06 \\
\texttt{qt(p\ =\ c(0.025,\ 0.975),\ df\ =\ 20-1)} & -2.09, 2.09 \\
\texttt{qt(p\ =\ c(0.025,\ 0.975),\ df\ =\ 16-1)} & -2.13, 2.13 \\
\bottomrule
\end{longtable}

\hypertarget{food-intake-and-weight-gain}{%
\section{Food intake and weight
gain}\label{food-intake-and-weight-gain}}

If we increase our food intake, we generally gain weight. Nutrition
scientists can calculate the amount of weight gain that would be
associated with a given increase in calories. In one study, 16 nonobese
adults, aged 25 to 36 years, were fed 1000 calories per day in excess of
the calories needed to maintain a stable body weight. The subjects
maintained this diet for 8 weeks, so they consumed a total of 56,000
extra calories. According to theory, 3500 extra calories will translate
into a weight gain of 1 pound. Therfore we expect each of these subjects
to gain 56,000/3500=16 pounds (lb). Here are the weights (given in the
\texttt{weightgain.csv} file) before and after the 8-week period
expressed in kilograms (kg):

\begin{Shaded}
\begin{Highlighting}[]
\NormalTok{weight }\OtherTok{\textless{}{-}} \FunctionTok{read.csv}\NormalTok{(}\StringTok{"weightgain.csv"}\NormalTok{)}
\end{Highlighting}
\end{Shaded}

\begin{enumerate}
\def\labelenumi{\alph{enumi}.}
\item
  Calculate a 95\% confidence interval for the mean weight change and
  give a sentence explaining the meaning of the 95\%. State your
  assumptions.
\item
  Calculate a 95\% bootstrap confidence interval for the mean weight
  change and compare it to the one obtained in part (a). Comment on the
  bootstrap sampling distribution and compare it to the assumptions you
  made in part (a).
\item
  Convert the units of the mean weight gain and 95\% confidence interval
  to pounds. Note that 1 kilogram is equal to 2.2 pounds. Test the null
  hypothesis that the mean weight gain is 16 lbs. State your assumptions
  and justify your choice of test. Be sure to specify the null and
  alternative hypotheses. What do you conclude?
\end{enumerate}

\newpage

\hypertarget{attitudes-toward-school}{%
\section{Attitudes toward school}\label{attitudes-toward-school}}

The Survey of Study Habits and Attitudes (SSHA) is a psychological test
that measures the motivation, attitude toward school, and study habits
of students. Scores range from 0 to 200. The mean score for U.S. college
students is about 115, and the standard deviation is about 30. A teacher
who suspects that older students have better attitudes toward school
gives the SSHA to 25 students who are at least 30 years of age. Their
mean score is \(\bar{y}\) = 132.2 with a sample standard deviation
\(s = 28\).

\begin{enumerate}
\def\labelenumi{\alph{enumi}.}
\tightlist
\item
  The teacher asks you to carry out a formal statistical test for her
  hypothesis. Perform a test, provide a 95\% confidence interval and
  state your conclusion clearly.
\item
  What assumptions did you use in part (a). Which of these assumptions
  is most important to the validity of your conclusion in part (a).
\end{enumerate}

\hypertarget{does-a-full-moon-affect-behavior}{%
\section{Does a full moon affect
behavior?}\label{does-a-full-moon-affect-behavior}}

Many people believe that the moon influences the actions of some
individuals. A study of dementia patients in nursing homes recorded
various types of disruptive behaviors every day for 12 weeks. Days were
classified as moon days if they were in a 3-day period centered at the
day of the full moon. For each patient, the average number of disruptive
behaviors was computed for moon days and for all other days. The
hypothesis is that moon days will lead to more disruptive behavior. We
look at a data set consisting of observations on 15 dementia patients in
nursing homes (available in the \texttt{fullmoon.csv} file):

\begin{Shaded}
\begin{Highlighting}[]
\NormalTok{fullmoon }\OtherTok{\textless{}{-}} \FunctionTok{read.csv}\NormalTok{(}\StringTok{"fullmoon.csv"}\NormalTok{)}
\end{Highlighting}
\end{Shaded}

\begin{enumerate}
\def\labelenumi{\alph{enumi}.}
\tightlist
\item
  Calculate a 95\% confidence interval for the mean difference in
  disruptive behaviors. State the assumptions you used to calculate this
  interval.
\item
  Test the hypothesis that moon days will lead to more disruptive
  behavior. State your assumptions and provide a brief conclusion based
  on your analysis.
\item
  Find the minimum value of the mean difference in disruptive behaviors
  (\(\bar{y}\)) needed to reject the null hypothesis.
\item
  What is the probability of detecting an increase of 1.0 aggressive
  behavior per day during moon days?
\end{enumerate}

%\showmatmethods


\bibliography{pinp}
\bibliographystyle{jss}



\end{document}
