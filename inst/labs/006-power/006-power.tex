\documentclass[letterpaper,11pt,twoside,]{pinp}

%% Some pieces required from the pandoc template
\providecommand{\tightlist}{%
  \setlength{\itemsep}{0pt}\setlength{\parskip}{0pt}}

% Use the lineno option to display guide line numbers if required.
% Note that the use of elements such as single-column equations
% may affect the guide line number alignment.

\usepackage[T1]{fontenc}
\usepackage[utf8]{inputenc}
\usepackage{longtable}

% pinp change: the geometry package layout settings need to be set here, not in pinp.cls
\geometry{layoutsize={0.95588\paperwidth,0.98864\paperheight},%
  layouthoffset=0.02206\paperwidth, layoutvoffset=0.00568\paperheight}

\definecolor{pinpblue}{HTML}{185FAF}  % imagecolorpicker on blue for new R logo
\definecolor{pnasbluetext}{RGB}{101,0,0} %



\title{Lab 006 - Power Calculations}

\author[a]{EPIB607 - Inferential Statistics}

  \affil[a]{McGill University}

\setcounter{secnumdepth}{5}

% Please give the surname of the lead author for the running footer
\leadauthor{Bhatnagar and Hanley}

% Keywords are not mandatory, but authors are strongly encouraged to provide them. If provided, please include two to five keywords, separated by the pipe symbol, e.g:
 

\begin{abstract}

\end{abstract}

\dates{This version was compiled on \today} 

% initially we use doi so keep for backwards compatibility
\doifooter{\url{https://sahirbhatnagar.com/EPIB607/}}
% new name is doi_footer

\pinpfootercontents{EPIB607}

\begin{document}

% Optional adjustment to line up main text (after abstract) of first page with line numbers, when using both lineno and twocolumn options.
% You should only change this length when you've finalised the article contents.
\verticaladjustment{-2pt}

\maketitle
\thispagestyle{firststyle}
\ifthenelse{\boolean{shortarticle}}{\ifthenelse{\boolean{singlecolumn}}{\abscontentformatted}{\abscontent}}{}

% If your first paragraph (i.e. with the \dropcap) contains a list environment (quote, quotation, theorem, definition, enumerate, itemize...), the line after the list may have some extra indentation. If this is the case, add \parshape=0 to the end of the list environment.


\begin{longtable}[]{@{}ll@{}}
\toprule
R Code & Value \\
\midrule
\endhead
\texttt{qnorm(p\ =\ c(0.05,\ 0.95))} & -1.64, 1.64 \\
\texttt{qnorm(p\ =\ c(0.025,\ 0.975))} & -1.96, 1.96 \\
\texttt{qnorm(p\ =\ c(0.005,\ 0.995))} & -2.58, 2.58 \\
\texttt{qt(p\ =\ c(0.025,\ 0.975),\ df\ =\ 400-1)} & -1.97, 1.97 \\
\texttt{qt(p\ =\ c(0.025,\ 0.975),\ df\ =\ 25-1)} & -2.06, 2.06 \\
\texttt{qt(p\ =\ c(0.025,\ 0.975),\ df\ =\ 20-1)} & -2.09, 2.09 \\
\texttt{qt(p\ =\ c(0.025,\ 0.975),\ df\ =\ 16-1)} & -2.13, 2.13 \\
\bottomrule
\end{longtable}

\hypertarget{lake-wobegon}{%
\section{Lake Wobegon}\label{lake-wobegon}}

It is claimed that the children of Lake Wobegon are above average. Take
a simple random sample of 9 children from Lake Wobegon, and measure
their IQ to obtain a sample mean of 112.8. IQ scores are scaled to be
Normally distributed with mean 100 and standard deviation 15.

\begin{enumerate}
\def\labelenumi{\alph{enumi})}
\item
  Does this sample provide evidence to reject the null hypothesis of no
  difference between children of Lake Wobegon and the general
  population?
\item
  Suppose you hope to use a one-sided test to show that the children
  from Lake Wobegon are at least 10 points higher than average on the IQ
  test. What power do you have to detect this with the sample of 9
  children if using a 0.05-level test?
\item
  If you hoped to use a \textbf{two-sided} test to show that the
  children from Lake Wobegon are at least 5 points higher than average
  on the IQ test, what power do you have with the sample size of 9 and a
  0.05-level test?
\end{enumerate}

\hypertarget{bias-in-step-counters}{%
\section{Bias in step counters}\label{bias-in-step-counters}}

Following the study by
\href{http://www.medicine.mcgill.ca/epidemiology/hanley/bios601/Surveys/SmartphoneSteps.pdf}{Case
et al., JAMA, 2015}, suppose we wished to assess, via a formal
statistical test, whether (at an \textit{population}, rather than an
individual, level) a step-counting device or app is unbiased (\(H_0\))
or under-counts (\(H_1\)). Suppose we will do so the way
\href{http://www.medicine.mcgill.ca/epidemiology/hanley/bios601/Surveys/SmartphoneSteps.pdf}{Case
et al.} did, but measuring \(n\) persons just once each. We observe the
device count when the true count on the treadmill reaches 500.

\begin{enumerate}
\def\labelenumi{\alph{enumi}.}
\tightlist
\item
  Using a planned sample size of \(n=25\), and \(\sigma = 60\) steps as
  a pre-study best-guess as to the \(s\) that might be observed in them,
  calculate the critical value at \(\alpha = 0.01\).
\item
  Now imagine that the mean would not be the null 500, but \(\mu=470.\)
  Calculate the probability that the mean in the sample of 25 will be
  less than this critical value. Use the same \(s\) for the alternative
  that you used for the null. What is this probability called?
\item
  Determine the sample size required for 80\% power using a 1\% level of
  significance. Plot the null and alternative distributions in a diagram
  using the
  \href{https://raw.githubusercontent.com/sahirbhatnagar/EPIB607/master/inst/code/plot_null_alt.R}{\texttt{plot\_power}}
  function.
\end{enumerate}

%\showmatmethods


\bibliography{pinp}
\bibliographystyle{jss}



\end{document}
