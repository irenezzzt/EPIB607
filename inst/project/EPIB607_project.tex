\documentclass[letterpaper,12pt,twoside,]{pinp}

%% Some pieces required from the pandoc template
\providecommand{\tightlist}{%
  \setlength{\itemsep}{0pt}\setlength{\parskip}{0pt}}

% Use the lineno option to display guide line numbers if required.
% Note that the use of elements such as single-column equations
% may affect the guide line number alignment.

\usepackage[T1]{fontenc}
\usepackage[utf8]{inputenc}

% pinp change: the geometry package layout settings need to be set here, not in pinp.cls
\geometry{layoutsize={0.95588\paperwidth,0.98864\paperheight},%
  layouthoffset=0.02206\paperwidth, layoutvoffset=0.00568\paperheight}

\definecolor{pinpblue}{HTML}{185FAF}  % imagecolorpicker on blue for new R logo
\definecolor{pnasbluetext}{RGB}{101,0,0} %



\title{Final Project}

\author[a]{EPIB607 - Inferential Statistics}

  \affil[a]{Fall 2021, McGill University}

\setcounter{secnumdepth}{5}

% Please give the surname of the lead author for the running footer
\leadauthor{Bhatnagar}

% Keywords are not mandatory, but authors are strongly encouraged to provide them. If provided, please include two to five keywords, separated by the pipe symbol, e.g:
 \keywords{  Final project  }  

\begin{abstract}
Final project instructions. Due date December 22, 2021.
\end{abstract}

\dates{This version was compiled on \today} 

% initially we use doi so keep for backwards compatibility
% new name is doi_footer

\pinpfootercontents{Final Project}

\begin{document}

% Optional adjustment to line up main text (after abstract) of first page with line numbers, when using both lineno and twocolumn options.
% You should only change this length when you've finalised the article contents.
\verticaladjustment{-2pt}

\maketitle
\thispagestyle{firststyle}
\ifthenelse{\boolean{shortarticle}}{\ifthenelse{\boolean{singlecolumn}}{\abscontentformatted}{\abscontent}}{}

% If your first paragraph (i.e. with the \dropcap) contains a list environment (quote, quotation, theorem, definition, enumerate, itemize...), the line after the list may have some extra indentation. If this is the case, add \parshape=0 to the end of the list environment.


\hypertarget{final-project}{%
\section{Final Project}\label{final-project}}

The objective of the group project is to construct an exercise and
solutions suitable for testing or demonstrating understanding of basic
principles of biostatistics as discussed in this course.

\vspace*{.3in}

Exercises must be based on (i) one to two articles in a scientific
journal or perhaps in the lay press or (ii) a
\textit{publicly available} dataset. The data must not be taken from an
RA project, but must be freely available on the web or another public
source. The article or data should concern some health problem amenable
to statistical investigation. The narrative of the exercise should be
clear and concise. The exercise should comprise 5-7 questions requiring
altogether about three hours for completion. The questions may cover any
part of this course. You must also produce a separate set of model
answers; these should be equally short and to the point.

\vspace*{.15in}

The group project will be evaluated using the following criteria (for a
total of 10 points):

\begin{enumerate}
\def\labelenumi{\arabic{enumi}.}
\tightlist
\item
  The choice of subject and ingenuity (2.0 points)\\
\item
  Testing of important biostatistical principles (2.5 points)\\
\item
  Exercises that are clear, concise, and creative. It's better to have
  one question that tests several concepts together, vs.~several
  questions that have no link with each other (2.5 points)\\
\item
  Quality of solutions (2.0 points)\\
\item
  Is the report reproducible (1.0 points)
\end{enumerate}

\vspace*{.1in}

\textbf{Projects should be done in groups of 2 to 4 people}. Examples
final projects prepared by students in previous years have been posted
on MyCourses. All projects must be uploaded to myCourses. One submission
per group.

The upload should consist of the following:\\
1. One \texttt{.Rmd} file containing the questions and solutions. This
must be fully reproducible using the techniques discussed in this class,
i.e., I should be able to download your submission, open the
\texttt{.Rmd} file, and compile it without error. Be aware of file paths
and hard coded solutions.\\
2. One compiled \texttt{.pdf} or \texttt{.html} file of the Rmarkdown
document\\
3. Any article(s) on which the questions are based\\
4. Any data-sets used in the questions, in text or CSV format. If the
dataset is publicly available then a link to the dataset or the R
package is sufficient.

%\showmatmethods


\bibliography{pinp}
\bibliographystyle{jss}



\end{document}
