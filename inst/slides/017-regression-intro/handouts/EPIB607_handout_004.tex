\documentclass[landscape,twocolumn,letterpaper,9pt,reqno]{article}\usepackage[]{graphicx}\usepackage[]{color}
%% maxwidth is the original width if it is less than linewidth
%% otherwise use linewidth (to make sure the graphics do not exceed the margin)
\makeatletter
\def\maxwidth{ %
  \ifdim\Gin@nat@width>\linewidth
    \linewidth
  \else
    \Gin@nat@width
  \fi
}
\makeatother

\definecolor{fgcolor}{rgb}{0.345, 0.345, 0.345}
\newcommand{\hlnum}[1]{\textcolor[rgb]{0.686,0.059,0.569}{#1}}%
\newcommand{\hlstr}[1]{\textcolor[rgb]{0.192,0.494,0.8}{#1}}%
\newcommand{\hlcom}[1]{\textcolor[rgb]{0.678,0.584,0.686}{\textit{#1}}}%
\newcommand{\hlopt}[1]{\textcolor[rgb]{0,0,0}{#1}}%
\newcommand{\hlstd}[1]{\textcolor[rgb]{0.345,0.345,0.345}{#1}}%
\newcommand{\hlkwa}[1]{\textcolor[rgb]{0.161,0.373,0.58}{\textbf{#1}}}%
\newcommand{\hlkwb}[1]{\textcolor[rgb]{0.69,0.353,0.396}{#1}}%
\newcommand{\hlkwc}[1]{\textcolor[rgb]{0.333,0.667,0.333}{#1}}%
\newcommand{\hlkwd}[1]{\textcolor[rgb]{0.737,0.353,0.396}{\textbf{#1}}}%
\let\hlipl\hlkwb

\usepackage{framed}
\makeatletter
\newenvironment{kframe}{%
 \def\at@end@of@kframe{}%
 \ifinner\ifhmode%
  \def\at@end@of@kframe{\end{minipage}}%
  \begin{minipage}{\columnwidth}%
 \fi\fi%
 \def\FrameCommand##1{\hskip\@totalleftmargin \hskip-\fboxsep
 \colorbox{shadecolor}{##1}\hskip-\fboxsep
     % There is no \\@totalrightmargin, so:
     \hskip-\linewidth \hskip-\@totalleftmargin \hskip\columnwidth}%
 \MakeFramed {\advance\hsize-\width
   \@totalleftmargin\z@ \linewidth\hsize
   \@setminipage}}%
 {\par\unskip\endMakeFramed%
 \at@end@of@kframe}
\makeatother

\definecolor{shadecolor}{rgb}{.97, .97, .97}
\definecolor{messagecolor}{rgb}{0, 0, 0}
\definecolor{warningcolor}{rgb}{1, 0, 1}
\definecolor{errorcolor}{rgb}{1, 0, 0}
\newenvironment{knitrout}{}{} % an empty environment to be redefined in TeX

\usepackage{alltt}

\usepackage{lscape,fancyhdr}

\usepackage{hyperref}

\pagestyle{fancy}

\usepackage{amsmath,epsfig,subfigure,amsthm,amsfonts,epsf,psfrag,rotating,setspace,bm}

\usepackage{verbatim,color} % Allow text colors}

\setlength{\oddsidemargin}{-0.4in}		% default=0in
\setlength\evensidemargin{-0.4in}

\setlength{\textwidth}{9.8in}		% default=9in

\setlength{\columnsep}{0.5in}		% default=10pt

\setlength{\columnseprule}{0pt}		% default=0pt (no line)


\setlength{\textheight}{7.0in}		% default=5.15in

\setlength{\topmargin}{-0.75in}		% default=0.20in

\setlength{\headsep}{0.25in}		% default=0.35in

\setlength{\parskip}{1.2ex}

\setlength{\parindent}{0mm}

\lhead{Course EPIB607: Regression handout 004 (Poisson regression)}
\rhead{jh,sb \ \ \ v. 2018.11.15}
\IfFileExists{upquote.sty}{\usepackage{upquote}}{}
\begin{document}
	


\section{Malaria control with bednets}

See the 2018 Lancet article \textit{Efficacy of Olyset Duo, a bednet containing pyriproxyfen and permethrin, versus a permethrin-only net against clinical malaria in an area with highly pyrethroid-resistant vectors in rural Burkina Faso: a cluster-randomised
	controlled trial} (\texttt{Bednets.pdf} in \texttt{A9} folder of myCourses) by Tiono et. al. Reproduce the Rate ratio (95\% CI) in Table 2. Calculate the rate difference and 95\% CI comparing PPF-treated to Standard long-lasting insecticidal nets. Check the goodness of fit. 




\begin{knitrout}
\definecolor{shadecolor}{rgb}{0.969, 0.969, 0.969}\color{fgcolor}
\begin{verbatim}
## 
## Call:
## glm(formula = cases ~ exposure + offset(log(years)), family = poisson(link = log), 
##     data = df)
## 
## Deviance Residuals: 
##     Min       1Q   Median       3Q      Max  
## -16.682   -4.732    1.497    3.984   12.024  
## 
## Coefficients:
##             Estimate Std. Error z value Pr(>|z|)    
## (Intercept)  0.68314    0.02432  28.092  < 2e-16 ***
## exposure    -0.26687    0.03286  -8.121 4.62e-16 ***
## ---
## Signif. codes:  0 '***' 0.001 '**' 0.01 '*' 0.05 '.' 0.1 ' ' 1
## 
## (Dispersion parameter for poisson family taken to be 1)
## 
##     Null deviance: 1381.2  on 23  degrees of freedom
## Residual deviance: 1316.0  on 22  degrees of freedom
## AIC: 1476.7
## 
## Number of Fisher Scoring iterations: 5
\end{verbatim}

\end{knitrout}
	




\clearpage


\section{Population mortality rates in Denmark}
\small 

\vspace*{-.1in}

We can fit the following simple (multiplicative) rate ratio model to the 
patterns of mortality rates  for 1980-1984 and  2000-2004. The reference cell is females 70-74,   1980-84. $R$ = rate. $M$ = multiplier.

\begin{tabular}{|l c | l l  l  l  | l l l l | l |}
	\hline
	Year  & Age & \multicolumn{3}{c}{Female (F)} & &   \multicolumn{3}{c}{Male (M)} & \\ 
	\hline
	& 70-74 &  $R_{F}$ & & & & $R_{F}$ & & $\times M_{M}$  & \\
	1980- & 75-79 &  $R_{F}$ & $ \times M_{75}$ & &   & $R_{F}$ & $\times M_{75}$ & $\times M_{M}$ & \\
	1984 & 80-84 & $R_{F}$ & $ \times M_{80}$ &  & &  $R_{F}$ & $ \times M_{80}$ & $ \times M_{M}$ & \\
	& 85-89 & $R_{F}$ & $ \times M_{85}$ &  & &  $R_{F}$ & $ \times M_{85}$ & $ \times M_{M}$ & \\ 
	\hline
	& 70-74 &  $R_{F}$ &  & & $\times M_{20y}$  &  $R_{F}$ & & $  \times M_{M}$  & $\times M_{20y}$\\
	2000- & 75-79 &  $R_{F} $ & $\times M_{75}$ & & $\times M_{20y}$ &  $R_{F}$ & $ \times M_{75}$ & $ \times M_{M}$& $\times M_{20y}$ \\
	2004      & 80-84 & $R_{F}$ & $ \times M_{80}$ & & $\times M_{20y}$ &   $R_{F}$ & $ \times M_{80}$ & $ \times M_{M}$ & $\times M_{20y}$ \\
	& 85-89 & $R_{F}$ & $ \times M_{85}$ & \ \ \ & $\times M_{20y}$&   $R_{F}$ & $\times M_{85}$ & $\times M_{M}$ & $\times M_{20y}$ \\
	\hline
\end{tabular}

%The array called `r' in the R code ( which fits additive models to the rates and logs of the rates)can be used to calculate ratios.

\begin{knitrout}
\definecolor{shadecolor}{rgb}{0.969, 0.969, 0.969}\color{fgcolor}
\begin{tabular}{l|l|r|r|r|r|r|r}
\hline
Year & Age & Female\_deaths & Female\_PT & Female\_rate & Male\_deaths & Male\_PT & Male\_rate\\
\hline
1980-1984 & 70-74 & 15989 & 586882.8 & 0.0272439 & 23810 & 456908.21 & 0.0521111\\
\hline
1980-1984 & 75-79 & 20838 & 454142.7 & 0.0458843 & 24707 & 300318.92 & 0.0822692\\
\hline
1980-1984 & 80-84 & 24073 & 297678.6 & 0.0808691 & 20319 & 167303.51 & 0.1214499\\
\hline
1980-1984 & 85-89 & 20216 & 147771.7 & 0.1368057 & 13524 & 74295.83 & 0.1820291\\
\hline
2000-2004 & 70-74 & 13912 & 521561.9 & 0.0266737 & 17360 & 436994.92 & 0.0397259\\
\hline
2000-2004 & 75-79 & 19731 & 471945.5 & 0.0418078 & 22477 & 341362.82 & 0.0658449\\
\hline
2000-2004 & 80-84 & 25541 & 369989.9 & 0.0690316 & 22992 & 217929.72 & 0.1055019\\
\hline
2000-2004 & 85-89 & 27135 & 226798.1 & 0.1196439 & 17444 & 104009.58 & 0.1677153\\
\hline
2005-2009 & 70-74 & 12179 & 540568.6 & 0.0225300 & 15782 & 472012.84 & 0.0334355\\
\hline
2005-2009 & 75-79 & 17273 & 444474.2 & 0.0388616 & 19547 & 344351.34 & 0.0567647\\
\hline
2005-2009 & 80-84 & 23513 & 363534.1 & 0.0646789 & 21781 & 230530.24 & 0.0944822\\
\hline
2005-2009 & 85-89 & 26842 & 237877.3 & 0.1128397 & 17811 & 114485.04 & 0.1555749\\
\hline
\end{tabular}


\end{knitrout}

%The equation is for the rate in any given age-group in a given gender in a given calendar period:
\textcolor{white}{text}\newline

\begin{tabular}{c c c c c c c c c}
	Rate = & $\rule{1cm}{0.15mm}$ & $\times \rule{1cm}{0.15mm}$ & $\times \rule{1cm}{0.15mm}$ & $\times \rule{1cm}{0.15mm}$ & $\times \rule{1cm}{0.15mm}$ & $\times \rule{1cm}{0.15mm}$ \\
	& &   if  &  if &  if & if & if & \\
	& &  75-79 & 80-84 & 85-89 & male & 2000-04 \\  \\
	$\log[Rate] =$ & $\rule{1cm}{0.15mm}$ & $+ \rule{1cm}{0.15mm}$ & $+ \rule{1cm}{0.15mm}$ & $+ \rule{1cm}{0.15mm}$ & $+ \rule{1cm}{0.15mm}$ & $+ \rule{1cm}{0.15mm}$ \\
	& &   if  &  if &  if & if & if & \\
	& &  75-79 & 80-84 & 85-89 & male & 2000-04 \\ \\
	
	$\log[Rate] =$ &$\rule{1cm}{0.15mm}$& $+  \rule{1cm}{0.15mm}$ & $+   \rule{1cm}{0.15mm}$ & $+   \rule{1cm}{0.15mm}$ & $+   \rule{1cm}{0.15mm} $ & $+ \rule{1cm}{0.15mm}$ \\
	& &  $\times$  &  $\times$ &  $\times$ & $\times$ & $\times$ & \\
	& &  $I_{75-79}$ & $I_{80-84}$ & $I_{85-89}$ & $I_{male}$ & $I_{2000-04}$ \\
\end{tabular}

where each $`I'$ is a (0/1) indicator of the category in question. By using both the 0 and 1 values of each $I$, this 6-parameter equation  produces a fitted value for each of the $4\times2\times2=16$ cells.

%You can also think of $I_{75-79},$  $I_{80-84},$ and  $I_{85-89}$ as 
%`radio buttons':  at most 1 of them can be `on' at the same time, since there are 4 
%age levels in all.


	
\end{document}
