\documentclass[landscape,twocolumn,letterpaper,9pt,reqno]{article}\usepackage[]{graphicx}\usepackage[]{color}
%% maxwidth is the original width if it is less than linewidth
%% otherwise use linewidth (to make sure the graphics do not exceed the margin)
\makeatletter
\def\maxwidth{ %
  \ifdim\Gin@nat@width>\linewidth
    \linewidth
  \else
    \Gin@nat@width
  \fi
}
\makeatother

\definecolor{fgcolor}{rgb}{0.345, 0.345, 0.345}
\newcommand{\hlnum}[1]{\textcolor[rgb]{0.686,0.059,0.569}{#1}}%
\newcommand{\hlstr}[1]{\textcolor[rgb]{0.192,0.494,0.8}{#1}}%
\newcommand{\hlcom}[1]{\textcolor[rgb]{0.678,0.584,0.686}{\textit{#1}}}%
\newcommand{\hlopt}[1]{\textcolor[rgb]{0,0,0}{#1}}%
\newcommand{\hlstd}[1]{\textcolor[rgb]{0.345,0.345,0.345}{#1}}%
\newcommand{\hlkwa}[1]{\textcolor[rgb]{0.161,0.373,0.58}{\textbf{#1}}}%
\newcommand{\hlkwb}[1]{\textcolor[rgb]{0.69,0.353,0.396}{#1}}%
\newcommand{\hlkwc}[1]{\textcolor[rgb]{0.333,0.667,0.333}{#1}}%
\newcommand{\hlkwd}[1]{\textcolor[rgb]{0.737,0.353,0.396}{\textbf{#1}}}%
\let\hlipl\hlkwb

\usepackage{framed}
\makeatletter
\newenvironment{kframe}{%
 \def\at@end@of@kframe{}%
 \ifinner\ifhmode%
  \def\at@end@of@kframe{\end{minipage}}%
  \begin{minipage}{\columnwidth}%
 \fi\fi%
 \def\FrameCommand##1{\hskip\@totalleftmargin \hskip-\fboxsep
 \colorbox{shadecolor}{##1}\hskip-\fboxsep
     % There is no \\@totalrightmargin, so:
     \hskip-\linewidth \hskip-\@totalleftmargin \hskip\columnwidth}%
 \MakeFramed {\advance\hsize-\width
   \@totalleftmargin\z@ \linewidth\hsize
   \@setminipage}}%
 {\par\unskip\endMakeFramed%
 \at@end@of@kframe}
\makeatother

\definecolor{shadecolor}{rgb}{.97, .97, .97}
\definecolor{messagecolor}{rgb}{0, 0, 0}
\definecolor{warningcolor}{rgb}{1, 0, 1}
\definecolor{errorcolor}{rgb}{1, 0, 0}
\newenvironment{knitrout}{}{} % an empty environment to be redefined in TeX

\usepackage{alltt}

\usepackage{lscape,fancyhdr}

\usepackage{hyperref}

\pagestyle{fancy}

\usepackage{amsmath,epsfig,subfigure,amsthm,amsfonts,epsf,psfrag,rotating,setspace,bm}

\usepackage{verbatim,color} % Allow text colors}

\setlength{\oddsidemargin}{-0.4in}		% default=0in
\setlength\evensidemargin{-0.4in}

\setlength{\textwidth}{9.8in}		% default=9in

\setlength{\columnsep}{0.5in}		% default=10pt

\setlength{\columnseprule}{0pt}		% default=0pt (no line)


\setlength{\textheight}{7.0in}		% default=5.15in

\setlength{\topmargin}{-0.75in}		% default=0.20in

\setlength{\headsep}{0.25in}		% default=0.35in

\setlength{\parskip}{1.2ex}

\setlength{\parindent}{0mm}

\lhead{Course EPIB607: Regression handout 005 (Logistic regression)}
\rhead{jh,sb \ \ \ v. 2018.11.19}
\IfFileExists{upquote.sty}{\usepackage{upquote}}{}
\begin{document}
	


\section{Kidney stone removal procedures 1}
The 1986 BMJ article \textit{Comparison of treatment of renal calculi by open surgery, percutaneous nephrolithotomy, and extracorporeal shockwave lithotripsy} by Charig et. al, was a study designed to compare different methods of treating kidney stones in order to establish which was the most cost effective and successful. The procedure, either open surgery, or percutaneous nephrolithotomy (PN, a keyhole surgery procedure), was defined to be successful if stones were eliminated or reduced to less than 2 mm after three months. The study collected cases of kidney stones treated at a particular UK hospital during 1972-1985. The counts of successes for the two surgical procedures were:

\begin{table}[h]
\centering
\begin{tabular}{lcc|c}
	& Unsuccessful &  Successful & Total\\
Open surgery & 77 & 273 & 350 \\
PN & 61 & 289 & 350 \\
\hline
Total & 138 & 562 & 700
\end{tabular}
\end{table}


\begin{knitrout}
\definecolor{shadecolor}{rgb}{0.969, 0.969, 0.969}\color{fgcolor}
\begin{verbatim}
## 
## Call:
## glm(formula = cbind(cases, controls) ~ open, family = binomial(link = "logit"), 
##     data = df)
## 
## Deviance Residuals: 
## [1]  0  0
## 
## Coefficients:
##             Estimate Std. Error z value Pr(>|z|)    
## (Intercept)  -1.5556     0.1409 -11.040   <2e-16 ***
## open          0.2899     0.1911   1.517    0.129    
## ---
## Signif. codes:  0 '***' 0.001 '**' 0.01 '*' 0.05 '.' 0.1 ' ' 1
## 
## (Dispersion parameter for binomial family taken to be 1)
## 
##     Null deviance: 2.3148e+00  on 1  degrees of freedom
## Residual deviance: 9.9920e-15  on 0  degrees of freedom
## AIC: 15.696
## 
## Number of Fisher Scoring iterations: 3
\end{verbatim}

\end{knitrout}

\clearpage

\section{Kidney stone removal procedures 2}
Below are the same outcomes tabulated by the size of the kidney stone (smaller than 2cm/at least 2cm in diameter):

\begin{table}[h]
	\centering
	\begin{tabular}{lcc|c}
		$<$ 2cm & Unsuccessful &  Successful & Total\\
		Open surgery & 6 & 81 & 87 \\
		PN 			 & 36 & 234 & 270 \\
		\hline
		Total 	& 42 & 315 & 357 \\
		& & &  \\
				$\geq$ 2cm & Unsuccessful &  Successful & Total\\
		Open surgery & 71 & 192 & 263 \\
		PN 			 & 25 & 55 & 80 \\
		\hline
		Total 		& 96 & 247 & 343
	\end{tabular}
\end{table}


\begin{knitrout}
\definecolor{shadecolor}{rgb}{0.969, 0.969, 0.969}\color{fgcolor}
\begin{verbatim}
## 
## Call:
## glm(formula = cbind(cases, controls) ~ open + size, family = binomial(link = "logit"), 
##     data = df)
## 
## Deviance Residuals: 
##       1        2        3        4  
## -0.7636   0.3588   0.2756  -0.4695  
## 
## Coefficients:
##             Estimate Std. Error z value Pr(>|z|)    
## (Intercept)  -1.9366     0.1704 -11.361  < 2e-16 ***
## open         -0.3572     0.2291  -1.559    0.119    
## size          1.2606     0.2390   5.274 1.33e-07 ***
## ---
## Signif. codes:  0 '***' 0.001 '**' 0.01 '*' 0.05 '.' 0.1 ' ' 1
## 
## (Dispersion parameter for binomial family taken to be 1)
## 
##     Null deviance: 33.1239  on 3  degrees of freedom
## Residual deviance:  1.0082  on 1  degrees of freedom
## AIC: 26.355
## 
## Number of Fisher Scoring iterations: 3
\end{verbatim}

\end{knitrout}

\clearpage

\section{Diabetes cohort data 1}

\begin{table}[h]
	\centering
	\begin{tabular}{lcc|c}
		& Dead &  Censored & Total\\
		Type II & 218 & 326 & 544 \\
		Type I & 105 & 253 & 323 \\
		\hline
		Total & 323 & 579 & 902
	\end{tabular}
\end{table}


\begin{knitrout}
\definecolor{shadecolor}{rgb}{0.969, 0.969, 0.969}\color{fgcolor}
\begin{verbatim}
## 
## Call:
## glm(formula = cbind(cases, controls) ~ type, family = binomial(link = "logit"), 
##     data = df)
## 
## Deviance Residuals: 
## [1]  0  0
## 
## Coefficients:
##             Estimate Std. Error z value Pr(>|z|)    
## (Intercept)  -0.8794     0.1161  -7.576 3.58e-14 ***
## type          0.4770     0.1454   3.282  0.00103 ** 
## ---
## Signif. codes:  0 '***' 0.001 '**' 0.01 '*' 0.05 '.' 0.1 ' ' 1
## 
## (Dispersion parameter for binomial family taken to be 1)
## 
##     Null deviance: 1.0978e+01  on 1  degrees of freedom
## Residual deviance: 1.4033e-13  on 0  degrees of freedom
## AIC: 16.858
## 
## Number of Fisher Scoring iterations: 2
\end{verbatim}

\end{knitrout}



\clearpage

\section{Diabetes cohort data 2}

Below are the same outcomes tabulated by age:

\begin{table}[h]
	\centering
	\begin{tabular}{lcc|c}
		$\leq 40$ & Dead &  Censored & Total\\
		Type II & 0 & 15 & 15 \\
		Type I & 1 & 129 & 130 \\
		\hline
		Total & 1 & 144 & 145 \\
		& & & \\
		$> 40$ & Dead &  Censored & Total\\
		Type II & 218 & 311 & 529 \\
		Type I & 104 & 124 & 228 \\
				\hline
		Total & 322 & 435 & 757 \\ 
	\end{tabular}
\end{table}


\begin{knitrout}
\definecolor{shadecolor}{rgb}{0.969, 0.969, 0.969}\color{fgcolor}
\begin{verbatim}
## 
## Call:
## glm(formula = cbind(dead, censored) ~ type + age, family = binomial(link = "logit"), 
##     data = df)
## 
## Deviance Residuals: 
##        1         2         3         4  
## -0.41982   0.09091   0.00776  -0.01168  
## 
## Coefficients:
##             Estimate Std. Error z value Pr(>|z|)    
## (Intercept)  -4.9525     1.0036  -4.935 8.02e-07 ***
## type         -0.1816     0.1595  -1.139    0.255    
## age           4.7781     1.0108   4.727 2.28e-06 ***
## ---
## Signif. codes:  0 '***' 0.001 '**' 0.01 '*' 0.05 '.' 0.1 ' ' 1
## 
## (Dispersion parameter for binomial family taken to be 1)
## 
##     Null deviance: 133.81237  on 3  degrees of freedom
## Residual deviance:   0.18471  on 1  degrees of freedom
## AIC: 20.745
## 
## Number of Fisher Scoring iterations: 5
\end{verbatim}

\end{knitrout}



\end{document}
