\documentclass[letterpaper,12pt,twoside,]{pinp}

%% Some pieces required from the pandoc template
\providecommand{\tightlist}{%
  \setlength{\itemsep}{0pt}\setlength{\parskip}{0pt}}

% Use the lineno option to display guide line numbers if required.
% Note that the use of elements such as single-column equations
% may affect the guide line number alignment.

\usepackage[T1]{fontenc}
\usepackage[utf8]{inputenc}

% pinp change: the geometry package layout settings need to be set here, not in pinp.cls
\geometry{layoutsize={0.95588\paperwidth,0.98864\paperheight},%
  layouthoffset=0.02206\paperwidth, layoutvoffset=0.00568\paperheight}

\definecolor{pinpblue}{HTML}{185FAF}  % imagecolorpicker on blue for new R logo
\definecolor{pnasbluetext}{RGB}{101,0,0} %



\title{022 - Contingency Tables and Difference of Proportions}

\author[a]{EPIB607 - Inferential Statistics}

  \affil[a]{Fall 2020, McGill University}

\setcounter{secnumdepth}{5}

% Please give the surname of the lead author for the running footer
\leadauthor{Bhatnagar}

% Keywords are not mandatory, but authors are strongly encouraged to provide them. If provided, please include two to five keywords, separated by the pipe symbol, e.g:
 

\begin{abstract}

\end{abstract}

\dates{This version was compiled on \today} 

% initially we use doi so keep for backwards compatibility
% new name is doi_footer


\begin{document}

% Optional adjustment to line up main text (after abstract) of first page with line numbers, when using both lineno and twocolumn options.
% You should only change this length when you've finalised the article contents.
\verticaladjustment{-2pt}

\maketitle
\thispagestyle{firststyle}
\ifthenelse{\boolean{shortarticle}}{\ifthenelse{\boolean{singlecolumn}}{\abscontentformatted}{\abscontent}}{}

% If your first paragraph (i.e. with the \dropcap) contains a list environment (quote, quotation, theorem, definition, enumerate, itemize...), the line after the list may have some extra indentation. If this is the case, add \parshape=0 to the end of the list environment.


\hypertarget{inference-for-binomial-proportions}{%
\section{Inference for Binomial
Proportions}\label{inference-for-binomial-proportions}}

\hypertarget{hypothesis-testing-with}{%
\subsection{\texorpdfstring{Hypothesis Testing with
\texttt{prop.test()}}{Hypothesis Testing with }}\label{hypothesis-testing-with}}

The function \texttt{prop.test()} is used to conduct a hypothesis test
for a single proportion or for the difference of two proportions, under
the assumption that the sampling distribution for each sample proportion
is approximately normal.

The \textbf{\texttt{prop.test()}} function has the following generic
structure:

\begin{Shaded}
\begin{Highlighting}[]
\KeywordTok{prop.test}\NormalTok{(x, n, }\DataTypeTok{alternative =} \StringTok{"two.sided"}\NormalTok{, }\DataTypeTok{p =} \FloatTok{0.5}\NormalTok{, }\DataTypeTok{conf.level =} \FloatTok{0.95}\NormalTok{, }\DataTypeTok{correct =} \OtherTok{TRUE}\NormalTok{)}
\end{Highlighting}
\end{Shaded}

where \texttt{x} is the count of successes, \texttt{n} is the number of
trials, \texttt{alternative} specifies the form of the alternative
hypothesis, \texttt{p} is \(p_0\), and \texttt{conf.level} refers to the
confidence level. The argument for \texttt{alternative} can be either
\texttt{"two.sided"} (\(H_A: p \neq p_0\)), \texttt{"less"}
(\(H_A: p < p_0\)), or \texttt{"greater"} (\(H_A: p > p_0\). By default,
confidence level is set to 95\% and a two-sided alternative is tested.
To conduct the two-sample test, enter \texttt{x} and \texttt{n} as
vectors; i.e., enter the number of successes in each group and the
number of trials in each group.

By default, Yates' continuity correction is applied where possible
(\texttt{correct = TRUE}). The purpose of the continuity correction is
to adjust for the error introduced by using a continuous distribution
(the \(\chi^2\) distribution) to model discrete probabilities; the
correction is meant to protect from underestimating \(p\)-values when
sample sizes are small. It has been shown, however, that the correction
can be overly strict and contribute to Type II error. In modern
practice, exact tests like the binomial test and Fisher's test are used
when sample sizes are small.

The following example shows a hypothesis test for testing the one-sided
hypothesis that the proportion of patients who respond to combined
therapy with nivolumab and ipilimumab is greater than 0.30, using data
that 21 out of 52 patients experienced a response.

\begin{Shaded}
\begin{Highlighting}[]
\KeywordTok{prop.test}\NormalTok{(}\DataTypeTok{x =} \DecValTok{21}\NormalTok{, }\DataTypeTok{n =} \DecValTok{52}\NormalTok{, }\DataTypeTok{alternative =} \StringTok{"greater"}\NormalTok{, }\DataTypeTok{p =} \FloatTok{0.30}\NormalTok{, }\DataTypeTok{conf.level =} \FloatTok{0.95}\NormalTok{)}
\end{Highlighting}
\end{Shaded}

\begin{ShadedResult}
\begin{verbatim}
#  
#   1-sample proportions test with continuity correction
#  
#  data:  21 out of 52, null probability 0.3
#  X-squared = 2.1987, df = 1, p-value = 0.06906
#  alternative hypothesis: true p is greater than 0.3
#  95 percent confidence interval:
#   0.2906582 1.0000000
#  sample estimates:
#          p 
#  0.4038462
\end{verbatim}
\end{ShadedResult}

The output of \texttt{prop.test()} is organized as a list object, and so
specific pieces can be extracted using the dollar sign (\texttt{\$}) and
the name of the desired component. The following examples show the
\(p\)-value and the confidence interval being selectively output from
the example shown above.

\begin{Shaded}
\begin{Highlighting}[]
\KeywordTok{prop.test}\NormalTok{(}\DataTypeTok{x =} \DecValTok{21}\NormalTok{, }\DataTypeTok{n =} \DecValTok{52}\NormalTok{, }\DataTypeTok{alternative =} \StringTok{"greater"}\NormalTok{, }\DataTypeTok{p =} \FloatTok{0.30}\NormalTok{, }\DataTypeTok{conf.level =} \FloatTok{0.95}\NormalTok{)}\OperatorTok{$}\NormalTok{p.val}
\end{Highlighting}
\end{Shaded}

\begin{ShadedResult}
\begin{verbatim}
#  [1] 0.06906279
\end{verbatim}
\end{ShadedResult}

\begin{Shaded}
\begin{Highlighting}[]
\KeywordTok{prop.test}\NormalTok{(}\DataTypeTok{x =} \DecValTok{21}\NormalTok{, }\DataTypeTok{n =} \DecValTok{52}\NormalTok{, }\DataTypeTok{alternative =} \StringTok{"greater"}\NormalTok{, }\DataTypeTok{p =} \FloatTok{0.30}\NormalTok{, }\DataTypeTok{conf.level =} \FloatTok{0.95}\NormalTok{)}\OperatorTok{$}\NormalTok{conf.int}
\end{Highlighting}
\end{Shaded}

\begin{ShadedResult}
\begin{verbatim}
#  [1] 0.2906582 1.0000000
#  attr(,"conf.level")
#  [1] 0.95
\end{verbatim}
\end{ShadedResult}

The following example shows a hypothesis test for testing the one-sided
hypothesis that the proportion of American adults who have sleep trouble
is less than 0.40, using data from \texttt{nhanes.samp.adult}.

\begin{Shaded}
\begin{Highlighting}[]
\CommentTok{#load the data}
\KeywordTok{library}\NormalTok{(oibiostat)}
\KeywordTok{data}\NormalTok{(}\StringTok{"nhanes.samp.adult"}\NormalTok{)}

\KeywordTok{prop.test}\NormalTok{(}\KeywordTok{sum}\NormalTok{(nhanes.samp.adult}\OperatorTok{$}\NormalTok{SleepTrouble }\OperatorTok{==}\StringTok{ "Yes"}\NormalTok{), }
          \KeywordTok{length}\NormalTok{(nhanes.samp.adult}\OperatorTok{$}\NormalTok{SleepTrouble), }\DataTypeTok{alternative =} \StringTok{"less"}\NormalTok{, }\DataTypeTok{p =} \FloatTok{0.40}\NormalTok{)}
\end{Highlighting}
\end{Shaded}

\begin{ShadedResult}
\begin{verbatim}
#  
#   1-sample proportions test with continuity correction
#  
#  data:  sum(nhanes.samp.adult$SleepTrouble == "Yes") out of length(nhanes.samp.adult$SleepTrouble), null probability 0.4
#  X-squared = 9.4522, df = 1, p-value = 0.001055
#  alternative hypothesis: true p is less than 0.4
#  95 percent confidence interval:
#   0.0000000 0.3373014
#  sample estimates:
#          p 
#  0.2666667
\end{verbatim}
\end{ShadedResult}

The following example shows a hypothesis test for the difference in
population proportions of breast cancer deaths between women who
received annual mammograms and women who received standard physical
exams. Of the 44,925 women in the mammogram group, 500 died of breast
cancer; of the 44,910 women in the control group, 505 died of breast
cancer.

\begin{Shaded}
\begin{Highlighting}[]
\KeywordTok{prop.test}\NormalTok{(}\DataTypeTok{x =} \KeywordTok{c}\NormalTok{(}\DecValTok{500}\NormalTok{, }\DecValTok{505}\NormalTok{), }\DataTypeTok{n =} \KeywordTok{c}\NormalTok{(}\DecValTok{44925}\NormalTok{, }\DecValTok{44910}\NormalTok{), }\DataTypeTok{alternative =} \StringTok{"two.sided"}\NormalTok{, }
          \DataTypeTok{conf.level =} \FloatTok{0.95}\NormalTok{)}
\end{Highlighting}
\end{Shaded}

\begin{ShadedResult}
\begin{verbatim}
#  
#   2-sample test for equality of proportions with continuity correction
#  
#  data:  c(500, 505) out of c(44925, 44910)
#  X-squared = 0.01748, df = 1, p-value = 0.8948
#  alternative hypothesis: two.sided
#  95 percent confidence interval:
#   -0.001512853  0.001282751
#  sample estimates:
#      prop 1     prop 2 
#  0.01112966 0.01124471
\end{verbatim}
\end{ShadedResult}

The following example shows a hypothesis test for the difference in
population proportions of sleep trouble between American men and women,
using data from \texttt{nhanes.samp.adult}.

\begin{Shaded}
\begin{Highlighting}[]
\NormalTok{x1 =}\StringTok{ }\KeywordTok{sum}\NormalTok{(nhanes.samp.adult}\OperatorTok{$}\NormalTok{Gender }\OperatorTok{==}\StringTok{ "female"} \OperatorTok{&}\StringTok{ }\NormalTok{nhanes.samp.adult}\OperatorTok{$}\NormalTok{SleepTrouble }\OperatorTok{==}\StringTok{ "Yes"}\NormalTok{)}
\NormalTok{x2 =}\StringTok{ }\KeywordTok{sum}\NormalTok{(nhanes.samp.adult}\OperatorTok{$}\NormalTok{Gender }\OperatorTok{==}\StringTok{ "male"} \OperatorTok{&}\StringTok{ }\NormalTok{nhanes.samp.adult}\OperatorTok{$}\NormalTok{SleepTrouble }\OperatorTok{==}\StringTok{ "Yes"}\NormalTok{)}
\NormalTok{n1 =}\StringTok{ }\KeywordTok{length}\NormalTok{(nhanes.samp.adult}\OperatorTok{$}\NormalTok{Gender }\OperatorTok{==}\StringTok{ "female"}\NormalTok{)}
\NormalTok{n2 =}\StringTok{ }\KeywordTok{length}\NormalTok{(nhanes.samp.adult}\OperatorTok{$}\NormalTok{Gender }\OperatorTok{==}\StringTok{ "male"}\NormalTok{)}

\KeywordTok{prop.test}\NormalTok{(}\DataTypeTok{x =} \KeywordTok{c}\NormalTok{(x1, x2), }\DataTypeTok{n =} \KeywordTok{c}\NormalTok{(n1, n2), }\DataTypeTok{alternative =} \StringTok{"two.sided"}\NormalTok{, }\DataTypeTok{conf.level =} \FloatTok{0.90}\NormalTok{)}
\end{Highlighting}
\end{Shaded}

\begin{ShadedResult}
\begin{verbatim}
#  
#   2-sample test for equality of proportions with continuity correction
#  
#  data:  c(x1, x2) out of c(n1, n2)
#  X-squared = 0.80128, df = 1, p-value = 0.3707
#  alternative hypothesis: two.sided
#  90 percent confidence interval:
#   -0.03087407  0.11976296
#  sample estimates:
#     prop 1    prop 2 
#  0.1555556 0.1111111
\end{verbatim}
\end{ShadedResult}

\hypertarget{hypothesis-testing-with-1}{%
\subsection{\texorpdfstring{Hypothesis Testing with
\texttt{binom.test()}}{Hypothesis Testing with }}\label{hypothesis-testing-with-1}}

The function \texttt{binom.test()} is used to conduct a hypothesis test
for a single proportion based on exact binomial probabilities.

The \textbf{\texttt{binom.test()}} function has the following generic
structure:

\begin{Shaded}
\begin{Highlighting}[]
\KeywordTok{binom.test}\NormalTok{(x, n, }\DataTypeTok{alternative =} \StringTok{"two.sided"}\NormalTok{, }\DataTypeTok{p =} \FloatTok{0.5}\NormalTok{, }\DataTypeTok{conf.level =} \FloatTok{0.95}\NormalTok{)}
\end{Highlighting}
\end{Shaded}

where \texttt{x} is the count of successes, \texttt{n} is the number of
trials, \texttt{alternative} specifies the form of the alternative
hypothesis, \texttt{p} is \(p_0\), and \texttt{conf.level} refers to the
confidence level. The argument for \texttt{alternative} can be either
\texttt{"two.sided"} (\(H_A: p \neq p_0\)), \texttt{"less"}
(\(H_A: p < p_0\)), or \texttt{"greater"} (\(H_A: p > p_0\). By default,
confidence level is set to 95\% and a two-sided alternative is tested.

The following example shows a hypothesis test for testing the one-sided
hypothesis that the proportion of glioblastoma patients who respond to
Avastin is different from 0.05, using data that 24 out of 85 patients
experienced a response.

\begin{Shaded}
\begin{Highlighting}[]
\KeywordTok{binom.test}\NormalTok{(}\DataTypeTok{x =} \DecValTok{24}\NormalTok{, }\DataTypeTok{n =} \DecValTok{85}\NormalTok{, }\DataTypeTok{alternative =} \StringTok{"two.sided"}\NormalTok{, }\DataTypeTok{p =} \FloatTok{0.05}\NormalTok{, }\DataTypeTok{conf.level =} \FloatTok{0.95}\NormalTok{)}
\end{Highlighting}
\end{Shaded}

\begin{ShadedResult}
\begin{verbatim}
#  
#   Exact binomial test
#  
#  data:  24 and 85
#  number of successes = 24, number of trials = 85, p-value = 2.674e-12
#  alternative hypothesis: true probability of success is not equal to 0.05
#  95 percent confidence interval:
#   0.1900101 0.3904038
#  sample estimates:
#  probability of success 
#               0.2823529
\end{verbatim}
\end{ShadedResult}

Note how an attempt to use \texttt{prop.test()} in this setting produces
a warning that the \(\chi^2\) (i.e., Normal) approximation may be
incorrect.

\begin{Shaded}
\begin{Highlighting}[]
\KeywordTok{prop.test}\NormalTok{(}\DataTypeTok{x =} \DecValTok{24}\NormalTok{, }\DataTypeTok{n =} \DecValTok{85}\NormalTok{, }\DataTypeTok{alternative =} \StringTok{"two.sided"}\NormalTok{, }\DataTypeTok{p =} \FloatTok{0.05}\NormalTok{, }\DataTypeTok{conf.level =} \FloatTok{0.95}\NormalTok{)}
\end{Highlighting}
\end{Shaded}

\begin{ShadedResult}
\begin{verbatim}
#  Warning in prop.test(x = 24, n = 85, alternative = "two.sided", p = 0.05, : Chi-
#  squared approximation may be incorrect
\end{verbatim}
\end{ShadedResult}
\begin{ShadedResult}
\begin{verbatim}
#  
#   1-sample proportions test with continuity correction
#  
#  data:  24 out of 85, null probability 0.05
#  X-squared = 91.78, df = 1, p-value < 2.2e-16
#  alternative hypothesis: true p is not equal to 0.05
#  95 percent confidence interval:
#   0.1926329 0.3920210
#  sample estimates:
#          p 
#  0.2823529
\end{verbatim}
\end{ShadedResult}

\newpage

\hypertarget{inference-for-two-way-tables}{%
\section{Inference for Two-Way
Tables}\label{inference-for-two-way-tables}}

\hypertarget{chi2-distribution-functions}{%
\subsection{\texorpdfstring{\(\chi^2\) Distribution
Functions}{\textbackslash chi\^{}2 Distribution Functions}}\label{chi2-distribution-functions}}

The function \textbf{\texttt{pchisq()}} used to calculate
\(P(X \leq k)\) or \(P(X > k)\) has the generic structure

\begin{Shaded}
\begin{Highlighting}[]
\KeywordTok{pchisq}\NormalTok{(q, df, }\DataTypeTok{lower.tail =} \OtherTok{TRUE}\NormalTok{)}
\end{Highlighting}
\end{Shaded}

where \texttt{q} is \(k\) and \texttt{df} is the degrees of freedom. By
default, \textsf{R} calculates \(P(X \leq k)\). In order to compute
\(P(X > k)\), specify \texttt{lower.tail = FALSE}.

The following example shows how to calculate \(P(X \leq 29.5)\) and
\(P(X > 29.5)\) for \(X \sim \chi^2_{df = 20}\).

\begin{Shaded}
\begin{Highlighting}[]
\CommentTok{#probability X is less than (or equal to) 1.20}
\KeywordTok{pchisq}\NormalTok{(}\FloatTok{29.5}\NormalTok{, }\DataTypeTok{df =} \DecValTok{20}\NormalTok{)}
\end{Highlighting}
\end{Shaded}

\begin{ShadedResult}
\begin{verbatim}
#  [1] 0.9216293
\end{verbatim}
\end{ShadedResult}

\begin{Shaded}
\begin{Highlighting}[]
\CommentTok{#probability X is greater than 1.20}
\KeywordTok{pchisq}\NormalTok{(}\FloatTok{29.5}\NormalTok{, }\DataTypeTok{df =} \DecValTok{20}\NormalTok{, }\DataTypeTok{lower.tail =} \OtherTok{FALSE}\NormalTok{)}
\end{Highlighting}
\end{Shaded}

\begin{ShadedResult}
\begin{verbatim}
#  [1] 0.07837072
\end{verbatim}
\end{ShadedResult}

\vspace{0.5cm}

The function \textbf{\texttt{qchisq()}} used to identify the observation
that corresponds to a particular probability \(p\) has the generic
structure

\begin{Shaded}
\begin{Highlighting}[]
\KeywordTok{qchisq}\NormalTok{(p, df, }\DataTypeTok{lower.tail =} \OtherTok{TRUE}\NormalTok{)}
\end{Highlighting}
\end{Shaded}

where \texttt{p} is \(p\) and \texttt{df} is the degrees of freedom. By
default, \textsf{R} identifies the observation that corresponds to area
\(p\) in the lower tail (i.e., to the left). To identify the observation
with area \(p\) in the upper tail, specify \texttt{lower.tail = FALSE}.

The following example shows how to calculate the value of the
observation where there is 0.922 area to the left (and 0.078 area to the
right) on a \(\chi^2\) distribution with 20 degrees of freedom.

\begin{Shaded}
\begin{Highlighting}[]
\CommentTok{#identify X value}
\KeywordTok{qchisq}\NormalTok{(}\FloatTok{0.922}\NormalTok{, }\DataTypeTok{df =} \DecValTok{20}\NormalTok{)}
\end{Highlighting}
\end{Shaded}

\begin{ShadedResult}
\begin{verbatim}
#  [1] 29.52077
\end{verbatim}
\end{ShadedResult}

\begin{Shaded}
\begin{Highlighting}[]
\CommentTok{#probability X is greater than 1.20}
\KeywordTok{qchisq}\NormalTok{(}\FloatTok{0.078}\NormalTok{, }\DataTypeTok{df =} \DecValTok{20}\NormalTok{, }\DataTypeTok{lower.tail =} \OtherTok{FALSE}\NormalTok{)}
\end{Highlighting}
\end{Shaded}

\begin{ShadedResult}
\begin{verbatim}
#  [1] 29.52077
\end{verbatim}
\end{ShadedResult}

\newpage

\hypertarget{entering-data-tables}{%
\subsection{Entering Data Tables}\label{entering-data-tables}}

The use of the function \texttt{matrix()} to construct matrices was
previously introduced in the Chapter 1 Lab Notes. For clarity when
displaying contingency tables, the matrix dimensions can be labeled
using \textbf{\texttt{dimnames()}}. The first entry in
\texttt{dimnames()} labels the rows and the second entry labels the
columns.

The following example shows a matrix from the HIV study comparing
nevirapine (NVP) and lopinarvir (LPV) with labeled dimensions.

\begin{Shaded}
\begin{Highlighting}[]
\CommentTok{#enter the data}
\NormalTok{hiv.table =}\StringTok{ }\KeywordTok{matrix}\NormalTok{(}\KeywordTok{c}\NormalTok{(}\DecValTok{60}\NormalTok{, }\DecValTok{27}\NormalTok{, }\DecValTok{87}\NormalTok{, }\DecValTok{113}\NormalTok{),}
                   \DataTypeTok{nrow =} \DecValTok{2}\NormalTok{, }\DataTypeTok{ncol =} \DecValTok{2}\NormalTok{, }\DataTypeTok{byrow =}\NormalTok{ T)}

\CommentTok{#add labels}
\KeywordTok{dimnames}\NormalTok{(hiv.table) =}\StringTok{ }\KeywordTok{list}\NormalTok{(}\StringTok{"Outcome"}\NormalTok{ =}\StringTok{ }\KeywordTok{c}\NormalTok{(}\StringTok{"Virologic Failure"}\NormalTok{, }\StringTok{"Stable Disease"}\NormalTok{),}
                           \StringTok{"Drug"}\NormalTok{ =}\StringTok{ }\KeywordTok{c}\NormalTok{(}\StringTok{"NVP"}\NormalTok{, }\StringTok{"LPV"}\NormalTok{))}

\NormalTok{hiv.table}
\end{Highlighting}
\end{Shaded}

\begin{ShadedResult}
\begin{verbatim}
#                     Drug
#  Outcome             NVP LPV
#    Virologic Failure  60  27
#    Stable Disease     87 113
\end{verbatim}
\end{ShadedResult}

\hypertarget{hypothesis-testing-with-2}{%
\subsection{\texorpdfstring{Hypothesis Testing with
\texttt{chisq.test()}}{Hypothesis Testing with }}\label{hypothesis-testing-with-2}}

The \textbf{\texttt{chisq.test()}} has the following generic structure

\begin{Shaded}
\begin{Highlighting}[]
\KeywordTok{chisq.test}\NormalTok{(x, y, }\DataTypeTok{correct =} \OtherTok{TRUE}\NormalTok{)}
\end{Highlighting}
\end{Shaded}

where \texttt{x} is either a matrix or a vector, \texttt{y} is a vector,
and Yates' continuity correction is applied by default. If \texttt{x} is
a matrix, the argument \texttt{y} is ignored.

The following example shows a test of independence for treatment and
outcome in the HIV data.

\begin{Shaded}
\begin{Highlighting}[]
\KeywordTok{chisq.test}\NormalTok{(hiv.table)}
\end{Highlighting}
\end{Shaded}

\begin{ShadedResult}
\begin{verbatim}
#  
#   Pearson's Chi-squared test with Yates' continuity correction
#  
#  data:  hiv.table
#  X-squared = 14.733, df = 1, p-value = 0.0001238
\end{verbatim}
\end{ShadedResult}

The output of \texttt{chisq.test()} is organized as a list object, and
so specific pieces can be extracted using the dollar sign (\texttt{\$})
and the name of the desired component. The following examples show the
residuals and expected values being selectively output from the test
conducted above.

\begin{Shaded}
\begin{Highlighting}[]
\KeywordTok{chisq.test}\NormalTok{(hiv.table)}\OperatorTok{$}\NormalTok{residuals}
\end{Highlighting}
\end{Shaded}

\begin{ShadedResult}
\begin{verbatim}
#                     Drug
#  Outcome                   NVP       LPV
#    Virologic Failure  2.312824 -2.369939
#    Stable Disease    -1.525412  1.563082
\end{verbatim}
\end{ShadedResult}

\begin{Shaded}
\begin{Highlighting}[]
\KeywordTok{chisq.test}\NormalTok{(hiv.table)}\OperatorTok{$}\NormalTok{expected}
\end{Highlighting}
\end{Shaded}

\begin{ShadedResult}
\begin{verbatim}
#                     Drug
#  Outcome                   NVP      LPV
#    Virologic Failure  44.56098 42.43902
#    Stable Disease    102.43902 97.56098
\end{verbatim}
\end{ShadedResult}

The following example shows a test of independence for statin use and
educational level from \texttt{prevend.samp}, using both entry options.

\begin{Shaded}
\begin{Highlighting}[]
\CommentTok{#load the data}
\KeywordTok{data}\NormalTok{(}\StringTok{"prevend.samp"}\NormalTok{)}

\CommentTok{#use x, y format}
\KeywordTok{chisq.test}\NormalTok{(prevend.samp}\OperatorTok{$}\NormalTok{Statin, prevend.samp}\OperatorTok{$}\NormalTok{Education)}
\end{Highlighting}
\end{Shaded}

\begin{ShadedResult}
\begin{verbatim}
#  
#   Pearson's Chi-squared test
#  
#  data:  prevend.samp$Statin and prevend.samp$Education
#  X-squared = 19.054, df = 3, p-value = 0.0002665
\end{verbatim}
\end{ShadedResult}

\begin{Shaded}
\begin{Highlighting}[]
\CommentTok{#enter x as a table}
\NormalTok{statin.edu.table =}\StringTok{ }\KeywordTok{table}\NormalTok{(prevend.samp}\OperatorTok{$}\NormalTok{Statin, prevend.samp}\OperatorTok{$}\NormalTok{Education)}
\KeywordTok{chisq.test}\NormalTok{(statin.edu.table)}
\end{Highlighting}
\end{Shaded}

\begin{ShadedResult}
\begin{verbatim}
#  
#   Pearson's Chi-squared test
#  
#  data:  statin.edu.table
#  X-squared = 19.054, df = 3, p-value = 0.0002665
\end{verbatim}
\end{ShadedResult}

\hypertarget{hypothesis-testing-with-3}{%
\subsection{\texorpdfstring{Hypothesis Testing with
\texttt{fisher.test( )}}{Hypothesis Testing with }}\label{hypothesis-testing-with-3}}

The \textbf{\texttt{fisher.test()}} has the following generic structure

\begin{Shaded}
\begin{Highlighting}[]
\KeywordTok{fisher.test}\NormalTok{(x, y, }\DataTypeTok{correct =} \OtherTok{TRUE}\NormalTok{)}
\end{Highlighting}
\end{Shaded}

where \texttt{x} is either a matrix or a vector and \texttt{y} is a
vector. If \texttt{x} is a matrix, the argument \texttt{y} is ignored.

The following example shows a test of independence for treatment and
outcome in the \emph{C. difficile} fecal infusion study.

\begin{Shaded}
\begin{Highlighting}[]
\CommentTok{#enter the data}
\NormalTok{infusion.table =}\StringTok{ }\KeywordTok{matrix}\NormalTok{(}\KeywordTok{c}\NormalTok{(}\DecValTok{13}\NormalTok{, }\DecValTok{3}\NormalTok{, }\DecValTok{4}\NormalTok{, }\DecValTok{9}\NormalTok{), }\DataTypeTok{nrow =} \DecValTok{2}\NormalTok{, }\DataTypeTok{ncol =} \DecValTok{2}\NormalTok{, }\DataTypeTok{byrow =}\NormalTok{ T)}
\KeywordTok{dimnames}\NormalTok{(infusion.table) =}\StringTok{ }\KeywordTok{list}\NormalTok{(}\StringTok{"Outcome"}\NormalTok{ =}\StringTok{ }\KeywordTok{c}\NormalTok{(}\StringTok{"Cured"}\NormalTok{, }\StringTok{"Uncured"}\NormalTok{),}
                           \StringTok{"Drug"}\NormalTok{ =}\StringTok{ }\KeywordTok{c}\NormalTok{(}\StringTok{"Fecal Infusion"}\NormalTok{, }\StringTok{"Vancomycin"}\NormalTok{))}

\KeywordTok{fisher.test}\NormalTok{(infusion.table)}
\end{Highlighting}
\end{Shaded}

\begin{ShadedResult}
\begin{verbatim}
#  
#   Fisher's Exact Test for Count Data
#  
#  data:  infusion.table
#  p-value = 0.00953
#  alternative hypothesis: true odds ratio is not equal to 1
#  95 percent confidence interval:
#    1.373866 78.811505
#  sample estimates:
#  odds ratio 
#    8.848725
\end{verbatim}
\end{ShadedResult}

\hypertarget{relative-risk-and-odds-ratio-with}{%
\subsection{\texorpdfstring{Relative Risk and Odds Ratio with
\texttt{epitools}}{Relative Risk and Odds Ratio with }}\label{relative-risk-and-odds-ratio-with}}

The \texttt{epitools} package contains various useful calculators for
epidemiology, including functions to calculate relative risk and odds
ratios in two-way tables. First, install and load the package:

\begin{Shaded}
\begin{Highlighting}[]
\KeywordTok{install.packages}\NormalTok{(}\StringTok{"epitools"}\NormalTok{)}
\KeywordTok{library}\NormalTok{(epitools)}
\end{Highlighting}
\end{Shaded}

The package requires that the rows of the table contain information on
exposure (i.e., treatment) while the columns of the table contain
information on outcome (i.e., disease), where the first row specifies
the baseline treatment group and the second column specifies presence of
the disease outcome.

The following example shows the HIV data re-entered to be in the
preferred format, where the rows of the table specify the type of drug
treatment and the columns specify the outcome. Note that the second
column specifies virologic failure.

\begin{Shaded}
\begin{Highlighting}[]
\CommentTok{#enter the data}
\NormalTok{hiv.table =}\StringTok{ }\KeywordTok{matrix}\NormalTok{(}\KeywordTok{c}\NormalTok{(}\DecValTok{87}\NormalTok{, }\DecValTok{113}\NormalTok{, }\DecValTok{60}\NormalTok{, }\DecValTok{27}\NormalTok{),}
                   \DataTypeTok{nrow =} \DecValTok{2}\NormalTok{, }\DataTypeTok{ncol =} \DecValTok{2}\NormalTok{, }\DataTypeTok{byrow =}\NormalTok{ F)}

\CommentTok{#add labels}
\KeywordTok{dimnames}\NormalTok{(hiv.table) =}\StringTok{ }\KeywordTok{list}\NormalTok{(}\StringTok{"Drug"}\NormalTok{ =}\StringTok{ }\KeywordTok{c}\NormalTok{(}\StringTok{"NVP"}\NormalTok{, }\StringTok{"LPV"}\NormalTok{),}
                           \StringTok{"Outcome"}\NormalTok{ =}\StringTok{ }\KeywordTok{c}\NormalTok{(}\StringTok{"Stable Disease"}\NormalTok{, }\StringTok{"Virologic Failure"}\NormalTok{))}

\NormalTok{hiv.table}
\end{Highlighting}
\end{Shaded}

\begin{ShadedResult}
\begin{verbatim}
#       Outcome
#  Drug  Stable Disease Virologic Failure
#    NVP             87                60
#    LPV            113                27
\end{verbatim}
\end{ShadedResult}

\vspace{0.5cm}

The function \textbf{\texttt{riskratio()}} calculates the relative risk
and has the following generic structure

\begin{Shaded}
\begin{Highlighting}[]
\KeywordTok{riskratio}\NormalTok{(x, }\DataTypeTok{y =} \OtherTok{NULL}\NormalTok{, }\DataTypeTok{rev =} \StringTok{"neither"}\NormalTok{)}\OperatorTok{$}\NormalTok{measure}
\end{Highlighting}
\end{Shaded}

where \texttt{x} is either a matrix or a vector and \texttt{y} is a
vector; if \texttt{x} is a matrix, then \texttt{y} is ignored. The
argument \texttt{rev} can be either \texttt{"neither"}, \texttt{"rows"},
\texttt{"columns"}, or \texttt{"both"}, and will either leave the data
as-is, reverse the ordering of the rows, reverse the ordering of the
columns, or reverse the ordering of both. To specifically output the
estimated relative risk, use \texttt{\$measure}.

In the following example, the relative risk of virologic failure is
calculated, first with nevirapine as the baseline then with lopinarvir
as the baseline. The estimated RR for the baseline group appears as 1.
The RR of virologic failure comparing NVP to LPV is 2.12; the risk of
virologic failure for individuals treated with nevirapine is over twice
that of the risk for those treated with lopinarvir.

The RR can also be calculated in terms of the risk of `success' (ie.,
stable disease); for example, the RR of stable disease comparing LPV to
NVP is 1.37; the risk of stable disease for individuals treated with
lopinarvir is 1.37 times that of the risk for those treated with
nevirapine.\footnote{Note that all combinations are only shown to illustrate use of \texttt{rev}; relative risks are generally interpreted in terms of presence of disease.}

\begin{Shaded}
\begin{Highlighting}[]
\CommentTok{#calculate RR of failure with respect to NVP}
\KeywordTok{riskratio}\NormalTok{(hiv.table)}\OperatorTok{$}\NormalTok{measure}
\end{Highlighting}
\end{Shaded}

\begin{ShadedResult}
\begin{verbatim}
#       risk ratio with 95% C.I.
#  Drug  estimate     lower    upper
#    NVP   1.0000        NA       NA
#    LPV   0.4725 0.3196517 0.698436
\end{verbatim}
\end{ShadedResult}

\begin{Shaded}
\begin{Highlighting}[]
\CommentTok{#calculate RR of failure with respect to LPV}
\KeywordTok{riskratio}\NormalTok{(hiv.table, }\DataTypeTok{rev =} \StringTok{"rows"}\NormalTok{)}\OperatorTok{$}\NormalTok{measure}
\end{Highlighting}
\end{Shaded}

\begin{ShadedResult}
\begin{verbatim}
#       risk ratio with 95% C.I.
#  Drug  estimate   lower    upper
#    LPV 1.000000      NA       NA
#    NVP 2.116402 1.43177 3.128405
\end{verbatim}
\end{ShadedResult}

\begin{Shaded}
\begin{Highlighting}[]
\CommentTok{#calculate RR of success with respect to NVP}
\KeywordTok{riskratio}\NormalTok{(hiv.table, }\DataTypeTok{rev =} \StringTok{"columns"}\NormalTok{)}\OperatorTok{$}\NormalTok{measure}
\end{Highlighting}
\end{Shaded}

\begin{ShadedResult}
\begin{verbatim}
#       risk ratio with 95% C.I.
#  Drug  estimate    lower    upper
#    NVP 1.000000       NA       NA
#    LPV 1.363793 1.165901 1.595274
\end{verbatim}
\end{ShadedResult}

\begin{Shaded}
\begin{Highlighting}[]
\CommentTok{#calculate RR of success with respect to LPV}
\KeywordTok{riskratio}\NormalTok{(hiv.table, }\DataTypeTok{rev =} \StringTok{"both"}\NormalTok{)}\OperatorTok{$}\NormalTok{measure}
\end{Highlighting}
\end{Shaded}

\begin{ShadedResult}
\begin{verbatim}
#       risk ratio with 95% C.I.
#  Drug   estimate     lower     upper
#    LPV 1.0000000        NA        NA
#    NVP 0.7332491 0.6268517 0.8577056
\end{verbatim}
\end{ShadedResult}

The following example shows the use of \texttt{riskratio} with vectors;
specifically, vectors from a larger dataframe. The relative risk of
cardiovascular disease comparing smokers to non-smokers is 1.09; smokers
have a 9\% higher risk of cardiovascular disease compared to
non-smokers, as estimated from \texttt{prevend.samp}.

\begin{Shaded}
\begin{Highlighting}[]
\CommentTok{#load the data}
\KeywordTok{data}\NormalTok{(}\StringTok{"prevend.samp"}\NormalTok{)}

\CommentTok{#convert to factors}
\NormalTok{prevend.samp}\OperatorTok{$}\NormalTok{Smoking =}\StringTok{ }\KeywordTok{factor}\NormalTok{(prevend.samp}\OperatorTok{$}\NormalTok{Smoking, }\DataTypeTok{levels =} \KeywordTok{c}\NormalTok{(}\DecValTok{0}\NormalTok{, }\DecValTok{1}\NormalTok{),}
                             \DataTypeTok{labels =} \KeywordTok{c}\NormalTok{(}\StringTok{"SmokeNo"}\NormalTok{, }\StringTok{"SmokeYes"}\NormalTok{))}
\NormalTok{prevend.samp}\OperatorTok{$}\NormalTok{CVD =}\StringTok{ }\KeywordTok{factor}\NormalTok{(prevend.samp}\OperatorTok{$}\NormalTok{CVD, }\DataTypeTok{levels =} \KeywordTok{c}\NormalTok{(}\DecValTok{0}\NormalTok{, }\DecValTok{1}\NormalTok{),}
                          \DataTypeTok{labels =} \KeywordTok{c}\NormalTok{(}\StringTok{"CVDNo"}\NormalTok{, }\StringTok{"CVDYes"}\NormalTok{))}

\CommentTok{#calculate the relative risk of CVD with respect to not smoking}
\KeywordTok{riskratio}\NormalTok{(prevend.samp}\OperatorTok{$}\NormalTok{Smoking, prevend.samp}\OperatorTok{$}\NormalTok{CVD)}\OperatorTok{$}\NormalTok{measure}
\end{Highlighting}
\end{Shaded}

\begin{ShadedResult}
\begin{verbatim}
#            risk ratio with 95% C.I.
#  Predictor  estimate     lower  upper
#    SmokeNo  1.000000        NA     NA
#    SmokeYes 1.086207 0.5476699 2.1543
\end{verbatim}
\end{ShadedResult}

\begin{Shaded}
\begin{Highlighting}[]
\CommentTok{#view the data table}
\KeywordTok{riskratio}\NormalTok{(prevend.samp}\OperatorTok{$}\NormalTok{Smoking, prevend.samp}\OperatorTok{$}\NormalTok{CVD)}\OperatorTok{$}\NormalTok{data}
\end{Highlighting}
\end{Shaded}

\begin{ShadedResult}
\begin{verbatim}
#            Outcome
#  Predictor  CVDNo CVDYes Total
#    SmokeNo    348     30   378
#    SmokeYes   106     10   116
#    Total      454     40   494
\end{verbatim}
\end{ShadedResult}

The function \textbf{\texttt{oddsratio()}} calculates the odds ratio and
has the following generic structure

\begin{Shaded}
\begin{Highlighting}[]
\KeywordTok{oddsratio}\NormalTok{(x, }\DataTypeTok{y =} \OtherTok{NULL}\NormalTok{, }\DataTypeTok{rev =} \StringTok{"neither"}\NormalTok{, }\DataTypeTok{method =} \StringTok{"wald"}\NormalTok{)}\OperatorTok{$}\NormalTok{measure}
\end{Highlighting}
\end{Shaded}

where \texttt{x} is either a matrix or a vector and \texttt{y} is a
vector; if \texttt{x} is a matrix, then \texttt{y} is ignored. The
argument \texttt{rev} can be either \texttt{"neither"}, \texttt{"rows"},
\texttt{"columns"}, or \texttt{"both"}, and will either leave the data
as-is, reverse the ordering of the rows, reverse the ordering of the
columns, or reverse the ordering of both. The argument \texttt{"wald"}
specifies that the odds ratio should be calculated using unconditional
maximum likelihood estimation; this corresponds to the formula used in
\emph{OI Biostat}. To specifically output the estimated odds, use
\texttt{\$measure}.

In the following example, the odds ratio of virologic failure is
calculated, first with nevirapine as the baseline then with lopinarvir
as the baseline. The OR for the baseline group appears as 1. The OR of
virologic failure comparing NVP to LPV is 2.87; the odds of virologic
failure for individuals treated with NVP are over twice as large as the
odds of failure for those treated with LPV.

\begin{Shaded}
\begin{Highlighting}[]
\CommentTok{#calculate OR of failure with respect to NVP}
\KeywordTok{oddsratio}\NormalTok{(hiv.table, }\DataTypeTok{method =} \StringTok{"wald"}\NormalTok{)}\OperatorTok{$}\NormalTok{measure}
\end{Highlighting}
\end{Shaded}

\begin{ShadedResult}
\begin{verbatim}
#       odds ratio with 95% C.I.
#  Drug   estimate     lower    upper
#    NVP 1.0000000        NA       NA
#    LPV 0.3464602 0.2032484 0.590581
\end{verbatim}
\end{ShadedResult}

\begin{Shaded}
\begin{Highlighting}[]
\CommentTok{#calculate OR of failure with respect to LPV}
\KeywordTok{oddsratio}\NormalTok{(hiv.table, }\DataTypeTok{rev =} \StringTok{"rows"}\NormalTok{, }\DataTypeTok{method =} \StringTok{"wald"}\NormalTok{)}\OperatorTok{$}\NormalTok{measure}
\end{Highlighting}
\end{Shaded}

\begin{ShadedResult}
\begin{verbatim}
#       odds ratio with 95% C.I.
#  Drug  estimate    lower    upper
#    LPV 1.000000       NA       NA
#    NVP 2.886335 1.693248 4.920088
\end{verbatim}
\end{ShadedResult}

The following example shows the use of \texttt{oddsratio} with vectors;
specifically, vectors from a larger dataframe. The odds ratio of
cardiovascular disease comparing smokers to non-smokers is 1.10, as
estimated from \texttt{prevend.samp}; the odds of cardiovascular disease
in smokers are 10\% larger than in non-smokers.

\begin{Shaded}
\begin{Highlighting}[]
\CommentTok{#calculate the odds ratio of CVD with respect to not smoking}
\KeywordTok{oddsratio}\NormalTok{(prevend.samp}\OperatorTok{$}\NormalTok{Smoking, prevend.samp}\OperatorTok{$}\NormalTok{CVD, }\DataTypeTok{method =} \StringTok{"wald"}\NormalTok{)}\OperatorTok{$}\NormalTok{measure}
\end{Highlighting}
\end{Shaded}

\begin{ShadedResult}
\begin{verbatim}
#            odds ratio with 95% C.I.
#  Predictor  estimate     lower    upper
#    SmokeNo   1.00000        NA       NA
#    SmokeYes  1.09434 0.5179749 2.312041
\end{verbatim}
\end{ShadedResult}

\begin{Shaded}
\begin{Highlighting}[]
\CommentTok{#view the data table}
\KeywordTok{oddsratio}\NormalTok{(prevend.samp}\OperatorTok{$}\NormalTok{Smoking, prevend.samp}\OperatorTok{$}\NormalTok{CVD)}\OperatorTok{$}\NormalTok{data}
\end{Highlighting}
\end{Shaded}

\begin{ShadedResult}
\begin{verbatim}
#            Outcome
#  Predictor  CVDNo CVDYes Total
#    SmokeNo    348     30   378
#    SmokeYes   106     10   116
#    Total      454     40   494
\end{verbatim}
\end{ShadedResult}

%\showmatmethods


\bibliography{pinp}
\bibliographystyle{jss}



\end{document}

